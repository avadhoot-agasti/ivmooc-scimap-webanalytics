\section{Introduction} \label{intro}

Between the years 2005 and 2014, the \textit{Places and Spaces: Mapping Science} exhibit worked towards the goal of bringing maps of science to the general public. In the year 2015, however, \textit{Places and Spaces} made moves in a direction that marked both a continuation of and a development upon its past achievements. While the exhibit's first decade was mainly devoted to static maps of science, the second decade's mission is devoted to exploring the power and potential of macroscopes; which can be thought of as interactive tools to analyze complex, vast and slow phenomenons in the field of science. The website which acts as a source of information about the exhibit has a visitor base aross the globe and hosts a lot of informational content; videos, games and many science maps. The usage statistics of the website for the period of ten years from 2007 to 2017 are available through Webalyzer reports in HTML format. Visualizing all the information in these HTML files through an interactive dashboard would provide insights about the chnages in web traffic on the website after it was redesigned. Python was chosen as a tool to scrape the data from 120 HTML files and and seggregate it into a set of tables which would then be used to create visualizations. The visualizations include descriptive statistics of visitor demographics, page visits, content downloads; geospatial origin of visits; correlation between exhibit events and user activity.

\subsection{Most Popular Pages} \label{viztoppages}
In this section we intend to discover the kind of content that is being consumed by the users on the scimaps.org website. 
To attain an intuitive understanding of this we have broadly classified the website content into 6 categories:
Data, Document/Text, Images, Video, Web and Other.
\begin{figure}
\centering
\fbox{\includegraphics[width=\linewidth]{img/TopURls.png}}
\caption{Distribution of Content Accessed (2007 - 2017)}
\label{fig:TopURLs}
\end{figure}
Following are some take aways from our analysis of the user behavior on scimaps.org: 
\begin{itemize}
\item Over the years there has always been a decrease in the amount of data consumed/downloaded. 
\item There has been a steady rise in the amount of documents browsed/downloaded over the years. The monotonic relationship doesn’t seem to have any effect of the website redesign as we observe a minor rise. A drastic change could have implied the user’s interest. 
\item Looking at the user behavior for images on scimaps.org, we see a clear interest. The images/visualizations showcased in the year 2010 marked the peak. There was over 3200$\%$ increase in the images viewed and 670,000$\%$ increase in the image data downloaded. This feat has not been repeated since then but we can see that the users are generally enticed by the image content that is being posted on the website. Since the website revamp in 2015, there has been a great increase user interest for images at scimaps.org.
\item  Taking a look at the video/media consumed by the users, we see a substantial increase in the videos downloaded and visited during 2010. This was the same case with images. This establishes a fact that the content that the website hosted in 2010 was highly appreciated. Since then we see a negative trend in the video content consumption and the website redesign doesn’t seem to have addressed that issue.
\item The website redesign doesn’t seem to have positively affected the users web based content consumption. 
\item Looking at some of the miscellaneous content that the users been to consume at scimaps.org, we again see a huge rise in 2010. But, since then the downloads and visits seem to have declined. There has been a steady decline since 2012 till date, which indicates that the redesign seems to have no impact.
\item Looking at the distribution of user visit and download pattern, we observe a trend. Earlier during 2007 -14 we observe huge web content consumed which drastically changed after 2015. This has been substituted by image, document and miscellaneous content.
\end{itemize}
